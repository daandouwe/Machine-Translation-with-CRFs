\documentclass[11pt, a4paper]{article}


\title{NLP2 Report}
\usepackage{amsmath}
\usepackage{amssymb}
\usepackage[compact,explicit]{titlesec}
\usepackage{geometry}
\usepackage{amsmath,amsfonts,amsthm,bm}
\usepackage[noend]{algpseudocode}
\usepackage{algorithm}
\usepackage{tikz}
\usepackage{hyperref}
\usepackage{tikz}
\usetikzlibrary{bayesnet}
\usetikzlibrary{fit,positioning}

% \usepackage{CJKutf8}
% \AtBeginDvi{\input{zhwinfonts}}



\begin{document}
\maketitle
Empirical evaluation (max 3 points): Is the experimental setup sound/convincing? Are experimental findings presented in an organised and effective manner? Here we expected to learn about:
\begin{enumerate}
	\item the data, experimental setup, practical constraints
	\item training conditions (the various choices and hyperparameters discussed and justified)
	\item ablation studies if applicable
	\item test results
	\item a critical discussion of findings
\end{enumerate}

\section{the data, experimental setup, practical constraints}

\begin{enumerate}
	\item You have hyper-parameters: SGD learning rate schedule, regularisation strength. Investigate them. Plot likelihood and validation BLEU.
	\item You have features of 3 sorts: segmentation, lexical translation, word order. Conduct ablation experiments.
	\item Report BLEU1 (it gives an ideal of how well your model performs lexical selection), report BLEU4 (it gives an idea of lexical selection and word ord).
\end{enumerate}

\subsection{experimental setup}

To evaluate the model we use a Chinese-English corpus consisting of 44k sentence pairs \cite{takezawa}. This corpus has two advantages: the sentences are relatively short (at most 30 words, and most of them shorter than 10), and the test-set contain multiple reference-translations \cite{takezawa}


We use an adaptive scheme for the learning rate of SGD following \cite{bottou}, where we let the learning rate decay as an inverse exponential of the number of parameter-updates. That is, we define the learning $\delta_t$ at parameter-update $t$ to be
\begin{align*}
\delta_t = \delta_0 (1 + \gamma\delta_0 t)^{-1}
\end{align*}
where $\delta_0$ is the initial learning rate at $t=0$ and $\gamma$ and aditional hyperparameter to determined. See the plot \ref{fig:learning-rates} for the effect of different values of $\gamma$.

\subsection{Determining hyperparameter}
In order to determine the settings for the hyperparameters involved in the Stochastic Gradient Descent (SGD) we performed a series of experiments on a small selection of training-data (200 sentence-pairs). These concern the choice of $\delta_0$ and $\gamma$, the strength of the $L_2$ regularizer $\lambda$, and the size of the minibatch. Plots \ref{fig:plots-likelihood} report of the experiments to determine the choice of $\delta_0$ and $\gamma$ by the average likelihood of training-sentences over iterations, and plots \ref{fig:plots-bleu} by the evolution of the BLEU score of the predictions the training-set. The plots in \ref{fig:batchsize} report of the experiments performed to determine the right batch-size by tracking the evolution of the BLEU-score. 

In these plots you can clearly see that...

The final settings chosen based on the experiments are $\delta_0=10$, $\gamma=1.0$, $\lambda=1$ and a minibatch-size of 1 sentence.

\subsection{results} % (fold)
\label{sub:results}

In tabel \ref{table:results} we list the results of the 


Table \ref{table:hyperparams} lists the final choices of training




\begin{table}
\begin{tabular}{ |l| } 
\hline
\textbf{C:} Chinese here! \\
\textbf{R:} I'd like to reserve two twin rooms. \\
\textbf{V:} My like reserve two one pair people room. \\
\hline
\textbf{C:} Chinese here! \\
\textbf{R:} Do you do alterations? \\
\textbf{V:} You may alterations you? \\
\hline
\textbf{C:} Chinese here! \\
\textbf{R:} I'd like some crayfish. \\
\textbf{V:} My like like some smaller lobster. \\
\hline
\textbf{C:} Chinese here! \\
\textbf{R:} I have a sore pain here. \\
\textbf{V:} My here very hurts. \\
\hline
\textbf{C:} Chinese here! \\
\textbf{R:} My's happy, green sir. \\
\textbf{V:} Mary is not so old as henr .\\
\hline
\textbf{C:} Chinese here! \\
\textbf{R:} How much is the breakfast? \\
\textbf{V:} Breakfast much much? \\
\hline
\end{tabular}
\caption{Example Viterbi-translations (V) against a reference translation (R)}
\label{table:translations}
\end{table}

\begin{table}
\begin{tabular}{ |l| } 
\hline
\textbf{C:} Chinese here! \\
\textbf{R:} Can you change yen into dollars? \\
\textbf{V:} Please my japanese change into dollar. \\
\textbf{I:} Please to yen exchange into dollars. \\
\hline
\textbf{C:} Chinese here! \\
\textbf{R:} Well, it's not so serious. \\
\textbf{V:} Well, nothing big problem. \\
\textbf{I:} Well, nothing big problem. \\
\hline
\textbf{C:} Chinese here! \\
\textbf{R:} I'm on my way to chicago from minneapolis. \\
\textbf{V:} In from 'm 'm go to a I. \\
\textbf{I:} The from minneapolis minneapolis to chicago the 'm. \\
\hline 
\end{tabular}
\caption{Example Viterbi (V) and IBM1 heuristic translations (I) against a reference translation (R)}
\label{table:translations-2}
\end{table}


% ? a a a
% ? a a i
% ? a a the
% ? a i a
% ? a i i
% ? a i the
% ? a the a
% ? a the i
% ? a the the
% ? alter a a a
% ? alter a a i
% ? alter a a the
% ? alter a i a
% ? alter a i i
% ? alter a i the
% ? alter a the a
% ? alter a the i
% ? alter a the the
% ? alter can a a a
% \vdots
% ? alterations can i the the
% ? alterations can the a a
% ? alterations can the a i
% ? alterations can the a the
% ? alterations can the i a
% ? alterations can the i i
% ? alterations can the i the
% ? alterations can the the a
% ? alterations can the the i
% ? alterations can the the the
% ? alterations can you a a a
% ? alterations can you a a i
% ? alterations can you a a the
% ? alterations can you a i a
% ? alterations can you a i i
% ? alterations can you a i the
% ? alterations can you a the a
% ? alterations can you a the i
% ? alterations can you a the the
% ? alterations can you i a a



\begin{table}
{\tiny
\begin{tabular}{ |l| }
\hline
a alterations you ? you \\
i the i alter ? do \\
your ? you a the a \\
? you alter your a  \\
i a i alterations ? you \\
the the a do ? can \\
alterations ? you can you i i the \\
alterations you ? i the i \\
may ? alter you a the a \\
may do ? alter i a a \\
i ? you may \\
a can alter ? you your \\
i the i may alterations ? you \\
alter do ? can your the the the \\
alter ? you may your a the i \\
alterations you ? may  a \\
alter do ? you i i i \\
the the i your may do ? alterations \\
i  alter you ? can you \\
a the a may you ? alterations \\
a do ? alterations may you \\
the a i alterations ? can \\
? do may your the i i \\
your ? you can the a the \\
? you alterations can your the the a \\
can ? do alterations your the \\
alterations you ? your a a i \\
? do alterations can \\
i a the you ? alterations you \\
can ? do alterations your a a i \\
your can alterations do ? \\
the the i your may ? alter \\
a a a ? you alter may your \\
your ? you alterations can i a i \\
alter you ? can your the the the \\
? you alterations can i  \\
may ? alterations your a a \\
alterations you ? may a  \\
you ? may you i the a \\
i the i your alterations ? may \\
your ? do alter can i the the \\
your ? you can i a i \\
i the the ? you alter \\
? you your i \\
alter you ? your i the i \\
? do alter your the a a \\
i  alterations ? you may \\
? you may the the the \\
may do ? i a \\
\hline
\end{tabular}}
\caption{An impression of the massive search-space of possible translations for the sentence `CHINESE HERE'. Shown is a selection of 50 out of a total of 26.514 sentences in the target forest. Note for example that only 1.458 sentence ($<6\%$) have the question mark at the end of the sentence---the most modest requirement for a proper translation of the sentence. Viewed this way, it is quite an achievement that the trained model managed to selecte 'you may alterations you ?' as the viterbi translation.}
\label{table:derrivations}
\end{table}

% \begin{CJK*}{UTF8}{zhsong}
% 文章内容。
% \end{CJK*}

\begin{table}
\begin{center}
\begin{tabular}{ |l|c| } 
\hline
feature $k$ & $\phi_k$ \\ 
\hline
span:rhs:src-rc:1-2 &   8.754e-14 \\
span:rhs:src-rc:2   &   4.006e-20 \\
span:rhs:src-lc:0-2 &   3.390e-20 \\
span:rhs:src-lc:1-1 &   -0.514 \\
span:rhs:tgt-lc:0-0 &   -0.531 \\
type:translation    &   -1.387 \\
type:deletion       &   -1.387 \\
trans:个/one         &-4.916e-39 \\
trans:个/a           & -4.916e-39 \\
skip-bigram:将*你    & -4.916e-39 \\
trans:给/me          & -4.916e-39 \\
type:terminal       &   -2.085 \\
type:insertion      &   -2.085 \\
ins:i               &   -2.085 \\
span:rhs:src:0      &   -2.085 \\
span:lhs:tgt:2-3    &   -2.085 \\
ins:the             &   -2.085 \\
ins:a               &   -2.085 \\
span:lhs:tgt:1-1    &   -2.128 \\
span:lhs:tgt:1-2    &   -2.129 \\
span:lhs:tgt:3      &   -21.709 \\
span:lhs:tgt:0-3    &   -21.709 \\
\hline
\end{tabular}
\end{center}
\caption{A selection of high-scoring features from the trained weights vector}
\label{table:features}
\end{table}



\end{document}




